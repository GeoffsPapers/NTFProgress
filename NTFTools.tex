\documentclass[runningheads]{llncs}

\usepackage[T1]{fontenc}
\usepackage{graphicx}
\usepackage{hyperref}
\usepackage{color}
\usepackage{setspace}
\usepackage{verbatim}
\renewcommand\UrlFont{\color{blue}\rmfamily}

\begin{document}

\title{Progress in the Development of\\ Non-classical Logics in the TPTP World}
\titlerunning{The Non-classical TPTP World}

\author{Alexander Steen\inst{1}\orcidID{0000-0001-8781-9462} 
\and
Geoff Sutcliffe\inst{2}\orcidID{0000-0001-9120-3927}}
\authorrunning{A. Steen, G. Sutcliffe}
\institute{University of Greifswald, Germany \\
\email{alexander.steen@uni-greifswald.de}
% \url{https://www.alexandersteen.de} 
\and
University of Miami, USA \\
\email{geoff@cs.miami.edu}
% \url{https://www.cs.miami.edu/home/geoff/} 
}

\maketitle
%--------------------------------------------------------------------------------------------------
\begin{abstract}
The abstract should briefly summarize the contents of the paper in
150--250 words.

\keywords{First keyword  \and Second keyword \and Another keyword.}
\end{abstract}
%--------------------------------------------------------------------------------------------------
\section{Introduction}
\label{Introduction}

The TPTP World \cite{Sut17} is a well established infrastructure that supports research, 
development, and deployment of Automated Theorem Proving (ATP) systems.
The TPTP World includes the TPTP problem library,
% \cite{Sut09}, 
the TSTP solution library,
% \cite{Sut10}, 
standards for writing ATP problems and reporting ATP solutions,
% \cite{SS+06,Sut08-KEAPPA}, 
tools and services for processing ATP problems and solutions,
% \cite{Sut10}, 
and it supports the CADE ATP System Competition (CASC).
% \cite{Sut16}.
Various parts of the TPTP World have been deployed in a range of applications,
in both academia and industry.
The web page \href{https://www.tptp.org}{\tt www.tptp.org} provides access to all 
components.

The development of the TPTP World has until now focused mostly on classical logic, while many 
real-world applications of ATP often also require non-classical reasoning. 
These applications include artificial intelligence (e.g., knowledge representation, planning, 
multi-agent systems), philosophy (e.g., formal ethics, metaphysics), natural language semantics 
(e.g., generalized quantifiers, modalities), and computer science (e.g., software and hardware 
verification).
This paper describes the latest extension of the TPTP World, providing languages and
infrastructure for reasoning in non-classical logics \cite{Pri08,Gob01}.
In this paper the languages and infrastructure are exemplified using modal logics \cite{BBW06}.

\paragraph{Paper structure.}~\\
Section~\ref{TPTPLanguages} reviews the general structure of the TPTP languages and

%--------------------------------------------------------------------------------------------------
\section{The Non-classical TPTP Languages}
\label{TPTPLanguages}

The TPTP languages for first-order clause normal form (CNF) \cite{SS98-JAR}, full first-order 
form (FOF) \cite{Sut09}, typed-first order form (TFF) \cite{SS+12,BP13-TFF1}, and typed 
higher-order form (THF) \cite{SB10,KSR16} are by now well known and regularly documented.
An overview that is relevant to this paper is provided in \cite{SF+22}, and the detailed
syntax of the languages is given in an extended BNF\footnote{%
\href{https://www.tptp.org/TPTP/SyntaxBNF.html}{\tt www.tptp.org/TPTP/SyntaxBNF.html}
\label{BNF}} \cite{VS06}.
As a simple reminder, here is an example monomorphic typed extended first-order (TX0) annotated 
formula~\ldots
\[
\begin{minipage}{\textwidth}
\begin{verbatim}
    tff(leaf_knaves_lie,axiom,
        ! [I: inhabitant,S: $o] : 
          ( ( is_knave(I) & says(I,S) ) => ~ S ),
        file('PUZ081_8.p',knaves_lie),
        [description('Knaves always lie'), relevance(0.9)]).
\end{verbatim}
\end{minipage}
\]
TXF (which breaks down into the monomorphic TX0 and polymorphic TX1) provides the basis for the 
non-classical typed extended first-order form (NXF).

The non-classical typed extended first-order form (NXF) and non-classical typed higher-order 
form (NHF) languages are the TPTP languages for non-classical logics.
NXF and NHF add an interpreted connective form
{\tt \verb|{|\$}{\em name}{\tt \verb|}|}.
Example {\em name}s are {\tt box}, {\tt dia}, {\tt possible}, {\tt necessary},
{\tt obligatory}, {\tt permissible}, {\tt knows}, {\tt believes}, etc.
In NXF the non-classical connectives are applied in a mixed applied/functional style, with the 
connectives applied to a {\tt ()}ed list of arguments.
In NHF the non-classical connectives are applied in higher-order style.
Figure~\ref{NX0Example} shows example types and alethic modal logic formulae in NX0.
A {\em name} may optionally be parameterized to reflect more complex non-classical connectives, 
e.g., in epistemic logic where the agent of knowledge is an index,
{\tt \verb|{|\$knows(\#manuel)\verb|}| @ (nothing)}\footnote{%
As in 
\href{https://www.youtube.com/watch?v=ISD86-oM4Ow}{\tt www.youtube.com/watch?v=ISD86-oM4Ow}}.
There are also {\em short form} unary connectives for (unparameterised) {\tt \$box} and 
{\tt \$dia}: {\tt [.]} and {\tt <.>}, e.g., {\tt \verb|{|\$box\verb|}| @ p} can be written 
{\tt [.] p}.
The syntax for formulae in non-classical logics is captured in the extended BNF with
{\tt <nhf\_long\_connective>}, {\tt <nxf\_atom>}, {\tt <nxf\_long\_connective>},
{\tt <ntf\_short\_connective>}, and {\tt <tff\_logic\_defn>},
The NXF and NHF languages can be used to write problems that are not naturally expressed in
classical logics.
Figure~\ref{NX0Example} shows a NX0 example problem.

In non-classical logics the same language can be used for formulae while different logics are 
used for reasoning.
It is therefore necessary to provide \mbox{(meta-)} information that specifies the
logic to be used.
A new kind of TPTP annotated formula has been introduced for this, with the role \texttt{logic},
and a ``logic specification'' as the formula.
A logic specification consists of a defined logic (family) name identified with a list of 
properties.
Figure~\ref{NX0Example} includes a logic specification in NX0 (note, the property names have
been improved since their presentation in \cite{SF+22}).

\begin{figure}[htbp]
\small
\setstretch{0.9}
\begin{verbatim}
tff(semantics,logic,
    $alethic_modal ==
      [ $domains == $constant,
        $rigidity == $rigid,
        $locality == $local,
        $modalities == $modal_system_M ] ).

tff(person_decl,type,person: $tType).
tff(product_decl,type,product: $tType).
tff(alex_decl,type,alex: person).
tff(chris_decl,type,chris: person).
tff(leo_decl,type,leo: product).
tff(work_hard_decl,type,work_hard: (person * product) > $o).
tff(gets_rich_decl,type,gets_rich: person > $o).

tff(work_hard_to_get_rich,axiom,
    ! [P: person] :
      ( ? [R: product] : work_hard(P,R)
      => {$possible} @ ( gets_rich(P) ) ) ).

tff(not_all_get_rich,axiom,
    ~ ? [P: person] : ({$necessary} @ (gets_rich(P)) ) ).

tff(alex_works_on_leo,axiom,
    work_hard(alex,leo) ).

tff(chris_works_on_leo,axiom,
    work_hard(chris,leo) ).

tff(only_alex_gets_rich,conjecture,
    ( {$possible} @ (gets_rich(alex)) 
    & {$possible} @ (~ gets_rich(chris)) ) ).
\end{verbatim}
\caption{NX0 example}
\label{NX0Example}
\end{figure}

%--------------------------------------------------------------------------------------------------
\section{Non-classical Proof and Models}
\label{ProofsModels}

The TPTP format for derivations \cite{SS+06} can immediately be used for writing derivations in
non-classical logic.

The new TPTP format for interpretations \cite{SS+23-LPAR} can be used to write Kripke models.
The worlds of a Kripke interpretation have the defined type {\tt \$ki\_world}, and are known to 
be distinct.
The defined predicate {\tt \$ki\_world\_is} of type {\tt \$ki\_world~>~\$o} is used to recognize 
worlds.
The defined predicate {\tt \$ki\_access\-ible} of type {\tt (\$ki\_world~*~\$ki\_world)~>~\$o} is
used to specify the accessibility between worlds.
Finally, the defined constant {\tt \$ki\_local\_world} of type {\tt \$ki\_world} is the world in
which a problem's conjecture must be proved. 
The Tarskian interpretations within worlds also use the new TPTP format for interpretations,
with guards used to specify the worlds in which the Tarskian interpretation is used.
Figure~\ref{NX0Kripke} shows the worlds and the first world's interpretation for a Kripke
model in NX0, for the problem in Figure~\ref{NX0Example}.

\begin{figure}[htbp]
\small
\setstretch{0.9}
\begin{verbatim}
tff(person_decl,type,    person: $tType).
tff(product_decl,type,   product: $tType).
tff(alex_decl,type,      alex: person).
tff(chris_decl,type,     chris: person).
tff(leo_decl,type,       leo: product).
tff(work_hard_decl,type, work_hard: (person * product) > $o).
tff(gets_rich_decl,type, gets_rich: person > $o).

tff(w1_decl,type,w1:     $ki_world).
tff(w2_decl,type,w2:     $ki_world).
tff(d_person_type,type,  d_person: $tType).
tff(d2person_decl,type,  d2person: d_person > person ).
tff(d_alex_decl,type,    d_alex: d_person).
tff(d_chris_decl,type,   d_chris: d_person).
tff(d_product_type,type, d_product: $tType).
tff(d2product_decl,type, d2product: d_product > product ).
tff(d_leo_decl,type,     d_leo: d_product).

tff(leo_workers,interpretation,
    ( ( ! [W: $ki_world] : ( W = w1 | W = w2 )
      & $distinct(w1,w2)
      & $ki_local_world = w1
      & $ki_accessible(w1,w1)     %----Logic is M
      & $ki_accessible(w2,w2)
      & $ki_accessible(w1,w2) )
    & ( $ki_world_is(w1,
        ( ( ! [P: person] : ? [DP: d_person] : P = d2person(DP)
          & ! [DP: d_person] : ( DP = d_alex | DP = d_chris )
          & $distinct(d_alex,d_chris)
          & ? [DP: d_person] : ( DP = d_alex )
          & ? [DP: d_person] : ( DP = d_chris )
          & ! [DP1: d_person,DP2: d_person] : 
              ( d2person(DP1) = d2person(DP2) => DP1 = DP2 )
          & ! [P: product] : ? [DP: d_product] : P = d2product(DP)
          & ! [DP: d_product] : DP = d_leo
          & ? [DP: d_product] : DP = d_leo
          & ! [DP1: d_product,DP2: d_product] :
              ( d2product(DP1) = d2product(DP2) => DP1 = DP2 ) )
        & ( alex = d2person(d_alex)
          & chris = d2person(d_chris)
          & leo = d2product(d_leo) )
        & ( work_hard(d2person(d_alex),d2product(d_leo))
          & work_hard(d2person(d_chris),d2product(d_leo))
          & ~ gets_rich(d2person(d_alex))
          & gets_rich(d2person(d_chris)) ) ) ) )
\end{verbatim}
\caption{NX0 Kripke model example}
\label{NX0Kripke}
\end{figure}

At the time of writing there are no ATP systems that can output proofs or Kripke models
in TPTP format for non-classical logic problems (hopefully given in TPTP format).
See Section~\ref{ATPSystems} for more information about non-classical logic ATP systems' 
capabilities.

%--------------------------------------------------------------------------------------------------
\section{Inside the TPTP}
\label{InsideTPTP}

A core piece of TPTP infrastructure is the TPTP problem library of test problems for
ATP systems.
To start with, problems in monomodal normal modal logic are being collected, including problems 
from (the citations are just some examples)
books \cite{For94,FM98,Gir00,Sid10}, 
conference and journal papers \cite{Rei92,FH+98,Sto00,PN+21}, 
research activities \cite{RP+07}, 
and use cases \cite{BW14-ECAI,MR22}.

The headers of non-classical logic problems have been updated to include relevant information.
The {\tt Syntax} field includes the number of classical connectives in the long form, the 
number that are indexed, and the number of connectives in the short-form, e.g,
the {\tt Number of connectives} in the problem in Figure~\ref{NX0Example} is\ldots
\begin{verbatim}
    8 (   2   ~;   0   |;   1   &)
      (   0 <=>;   1  =>;   0  <=;   0 <~>)
      (   4 {}@;   0 {#};   0 {.})
\end{verbatim}
The {\tt SPC} field has values for non-classical logic problems, e.g., the SPC for the problem 
in Figure~\ref{NX0Example} is {\tt NX0\_THM\_NEQ\_NAR}.
This information is now part of the {\tt ProblemAndSolutionStatistics} file in the {\tt Documents}
directory of the TPTP distribution.

%--------------------------------------------------------------------------------------------------
\section{ATP Systems}
\label{ATPSystems}

      \begin{enumerate}
      \item Provers
            \begin{enumerate}
            \item  Native: Leo-III (but really it's via NTF2THF)
            \item Via NTF2THF: Any THF prover, e.g., Vampire, E, cvc5, Satallax, etc.
            \item Via TPTP2KSP: Claudia's tool, others?
            \end{enumerate}
      \item Model finders
            \begin{enumerate}
            \item Native: None (Leo-III in SAT mode does nothing)
            \item Via NTF2THF: Any THF model finder, e.g, Nitpick, cvc5, LEO-II, Satallax, Vampire
            \end{enumerate}
      \end{enumerate}

%--------------------------------------------------------------------------------------------------
\section{Service Tools}
\label{ServiceTools}

\begin{enumerate}
\item Utility tools
      \begin{enumerate}
      \item Parsers and printers: tptp-utils-app, TPTP4X
      \item Language transformers
      \end{enumerate}

\item Language transformers
      \begin{enumerate}
      \item NTF2THF - logic-embedding-app.jar
      \item NXF2TXF (coming from Alex)
      \item Short2Long (coming from Alex)
      \item TPTP2KSP and KSP2TPTP
      \end{enumerate}

\item Solution analysers
      \begin{enumerate}
      \item GDV - Geoff's Derivation Verifier (needs to be extended?)
      \item AGMV - Alex and Geoff's Model Verifier: Via TXF2TFF
      \item IDV - Interactive Derivation Verifier (for when we have NTF proof output)
      \item Alex's student's interpretation visualizer
      \end{enumerate}

\item Online services
      \begin{enumerate}
      \item TPTP2T
      \item System*T?TP
      \end{enumerate}

\end{enumerate}

%--------------------------------------------------------------------------------------------------
\section{Conclusion}
\label{Conclusion}

%--------------------------------------------------------------------------------------------------
\bibliographystyle{splncs04}
\bibliography{Bibliography}
%--------------------------------------------------------------------------------------------------
\end{document}
